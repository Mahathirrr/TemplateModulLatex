\begin{center}
    \addcontentsline{toc}{section}{Modul 2 Praktikum \matakuliah}
    \textbf{Modul 2 Praktikum \matakuliah}

    \textbf{Preprocessing Teks dan Tokenisasi}
\end{center}

\addcontentsline{toc}{subsection}{Deskripsi Materi}
\section*{Deskripsi Singkat}
Modul ini membahas tentang preprocessing teks dan tokenisasi sebagai langkah fundamental dalam sistem pencarian informasi. Preprocessing teks adalah proses membersihkan dan mempersiapkan data teks mentah agar dapat diproses lebih efektif oleh sistem IR.

\addcontentsline{toc}{subsection}{Tujuan Praktikum}
\section*{Tujuan}
\begin{enumerate}
    \item Memahami pentingnya preprocessing dalam sistem pencarian informasi
    \item Dapat melakukan tokenisasi pada teks bahasa Indonesia dan Inggris
    \item Dapat menerapkan case folding dan normalisasi teks
    \item Dapat menghilangkan stopwords yang tidak relevan
    \item Dapat menerapkan stemming untuk mengurangi variasi kata
    \item Memahami dampak preprocessing terhadap kualitas pencarian
\end{enumerate}

\addcontentsline{toc}{subsection}{Materi 1 - Pengenalan Preprocessing Teks}
\subsection*{Materi 1 - Pengenalan Preprocessing Teks}

Preprocessing teks adalah tahap penting dalam sistem pencarian informasi yang bertujuan untuk:

\textbf{Tujuan Preprocessing:}
\begin{itemize}
    \item \textbf{Mengurangi noise}: Menghilangkan karakter atau kata yang tidak relevan
    \item \textbf{Standardisasi}: Menyeragamkan format teks
    \item \textbf{Efisiensi}: Mengurangi ukuran data dan mempercepat pencarian
    \item \textbf{Akurasi}: Meningkatkan ketepatan hasil pencarian
\end{itemize}

\textbf{Tahapan Preprocessing Teks:}
\begin{enumerate}
    \item \textbf{Case Folding}: Mengubah semua huruf menjadi huruf kecil
    \item \textbf{Tokenisasi}: Memecah teks menjadi token (kata atau unit terkecil)
    \item \textbf{Filtering}: Menghilangkan karakter khusus dan tanda baca
    \item \textbf{Stopword Removal}: Menghilangkan kata-kata yang tidak bermakna
    \item \textbf{Stemming/Lemmatization}: Mengubah kata ke bentuk dasarnya
    \item \textbf{Normalisasi}: Menangani singkatan, angka, dan variasi penulisan
\end{enumerate}

\textbf{Contoh Sebelum dan Sesudah Preprocessing:}
\begin{itemize}
    \item \textbf{Sebelum}: "Saya SEDANG mencari-cari informasi tentang Universitas Syiah Kuala!!!"
    \item \textbf{Sesudah}: ["cari", "informasi", "universitas", "syiah", "kuala"]
\end{itemize}

\addcontentsline{toc}{subsection}{Materi 2 - Tokenisasi dengan NLTK}
\subsection*{Materi 2 - Tokenisasi dengan NLTK}

Tokenisasi adalah proses memecah teks menjadi unit-unit yang lebih kecil seperti kata, frasa, atau kalimat. NLTK (Natural Language Toolkit) adalah library Python yang powerful untuk pemrosesan bahasa alami.

\textbf{Instalasi Library:}
\begin{lstlisting}[language=bash, style=bash]
pip install nltk
pip install Sastrawi
\end{lstlisting}

\textbf{Setup NLTK:}
\begin{lstlisting}[language=python, style=python]
import nltk
from nltk.tokenize import word_tokenize, sent_tokenize
from nltk.corpus import stopwords
import string
import re

# Download resource NLTK yang diperlukan
nltk.download('punkt')
nltk.download('stopwords')
\end{lstlisting}

\textbf{Tokenisasi Kalimat:}
\begin{lstlisting}[language=python, style=python]
# Contoh teks
text = """Pencarian informasi adalah bidang yang menarik.
Sistem IR membantu kita menemukan informasi yang relevan.
Preprocessing adalah langkah penting dalam IR."""

# Tokenisasi kalimat
sentences = sent_tokenize(text)
print("Tokenisasi Kalimat:")
for i, sentence in enumerate(sentences, 1):
    print(f"{i}. {sentence.strip()}")

# Output:
# 1. Pencarian informasi adalah bidang yang menarik.
# 2. Sistem IR membantu kita menemukan informasi yang relevan.
# 3. Preprocessing adalah langkah penting dalam IR.
\end{lstlisting}

\textbf{Tokenisasi Kata:}
\begin{lstlisting}[language=python, style=python]
# Tokenisasi kata
text = "Sistem pencarian informasi modern menggunakan berbagai algoritma."
words = word_tokenize(text)
print("Tokenisasi Kata:")
print(words)

# Output:
# ['Sistem', 'pencarian', 'informasi', 'modern', 'menggunakan', 'berbagai', 'algoritma', '.']
\end{lstlisting}

\addcontentsline{toc}{subsection}{Materi 3 - Case Folding dan Normalisasi}
\subsection*{Materi 3 - Case Folding dan Normalisasi}

Case folding adalah proses mengubah semua huruf dalam teks menjadi huruf kecil untuk menghindari perbedaan yang tidak perlu antara kata yang sama.

\textbf{Case Folding Sederhana:}
\begin{lstlisting}[language=python, style=python]
def case_folding(text):
    """Mengubah teks menjadi huruf kecil"""
    return text.lower()

# Contoh penggunaan
text = "Universitas SYIAH KUALA adalah Universitas Terbaik"
folded_text = case_folding(text)
print("Sebelum:", text)
print("Sesudah:", folded_text)

# Output:
# Sebelum: Universitas SYIAH KUALA adalah Universitas Terbaik
# Sesudah: universitas syiah kuala adalah universitas terbaik
\end{lstlisting}

\textbf{Normalisasi Teks Lanjutan:}
\begin{lstlisting}[language=python, style=python]
def normalize_text(text):
    """Normalisasi teks lengkap"""
    # Case folding
    text = text.lower()

    # Hilangkan URL
    text = re.sub(r'http\S+|www\S+|https\S+', '', text)

    # Hilangkan email
    text = re.sub(r'\S+@\S+', '', text)

    # Hilangkan angka
    text = re.sub(r'\d+', '', text)

    # Hilangkan karakter khusus kecuali spasi
    text = re.sub(r'[^a-zA-Z\s]', '', text)

    # Hilangkan spasi berlebih
    text = re.sub(r'\s+', ' ', text).strip()

    return text

# Contoh penggunaan
text = "Kunjungi website kami di https://unsyiah.ac.id atau email ke info@unsyiah.ac.id. Telp: 0651-123456"
normalized = normalize_text(text)
print("Sebelum:", text)
print("Sesudah:", normalized)

# Output:
# Sebelum: Kunjungi website kami di https://unsyiah.ac.id atau email ke info@unsyiah.ac.id. Telp: 0651-123456
# Sesudah: kunjungi website kami di atau email ke telp
\end{lstlisting}

\addcontentsline{toc}{subsection}{Materi 4 - Stopword Removal}
\subsection*{Materi 4 - Stopword Removal}

Stopwords adalah kata-kata yang sering muncul dalam teks tetapi tidak memberikan informasi yang bermakna untuk pencarian, seperti "dan", "atau", "yang", "adalah", "di", "ke", dll.

\textbf{Menggunakan Stopwords NLTK:}
\begin{lstlisting}[language=python, style=python]
# Load stopwords bahasa Indonesia
indonesian_stopwords = set(stopwords.words('indonesian'))
print("Jumlah stopwords Indonesia:", len(indonesian_stopwords))
print("Contoh stopwords:", list(indonesian_stopwords)[:10])

# Load stopwords bahasa Inggris
english_stopwords = set(stopwords.words('english'))
print("Jumlah stopwords Inggris:", len(english_stopwords))
print("Contoh stopwords:", list(english_stopwords)[:10])
\end{lstlisting}

\textbf{Implementasi Stopword Removal:}
\begin{lstlisting}[language=python, style=python]
def remove_stopwords(tokens, language='indonesian'):
    """Menghilangkan stopwords dari list token"""
    if language == 'indonesian':
        stop_words = set(stopwords.words('indonesian'))
    else:
        stop_words = set(stopwords.words('english'))

    # Filter token yang bukan stopwords
    filtered_tokens = [token for token in tokens if token not in stop_words]
    return filtered_tokens

# Contoh penggunaan
text = "sistem pencarian informasi adalah bidang yang menarik dan penting"
tokens = word_tokenize(text.lower())
print("Sebelum stopword removal:", tokens)

filtered_tokens = remove_stopwords(tokens)
print("Sesudah stopword removal:", filtered_tokens)

# Output:
# Sebelum: ['sistem', 'pencarian', 'informasi', 'adalah', 'bidang', 'yang', 'menarik', 'dan', 'penting']
# Sesudah: ['sistem', 'pencarian', 'informasi', 'bidang', 'menarik', 'penting']
\end{lstlisting}

\textbf{Custom Stopwords:}
\begin{lstlisting}[language=python, style=python]
# Menambahkan stopwords custom
def create_custom_stopwords():
    """Membuat daftar stopwords custom"""
    base_stopwords = set(stopwords.words('indonesian'))

    # Tambahkan stopwords khusus domain
    custom_words = {
        'sistem', 'dapat', 'akan', 'telah', 'sudah',
        'sangat', 'lebih', 'juga', 'hanya', 'saja'
    }

    all_stopwords = base_stopwords.union(custom_words)
    return all_stopwords

# Penggunaan
custom_stopwords = create_custom_stopwords()
tokens = ['sistem', 'pencarian', 'informasi', 'dapat', 'membantu', 'pengguna']
filtered = [token for token in tokens if token not in custom_stopwords]
print("Dengan custom stopwords:", filtered)
\end{lstlisting}

\addcontentsline{toc}{subsection}{Materi 5 - Stemming dengan Sastrawi}
\subsection*{Materi 5 - Stemming dengan Sastrawi}

Stemming adalah proses mengubah kata berimbuhan menjadi kata dasar. Untuk bahasa Indonesia, kita menggunakan library Sastrawi yang mengimplementasikan algoritma stemming khusus bahasa Indonesia.

\textbf{Setup Sastrawi:}
\begin{lstlisting}[language=python, style=python]
from Sastrawi.Stemmer.StemmerFactory import StemmerFactory

# Membuat stemmer
factory = StemmerFactory()
stemmer = factory.create_stemmer()

# Contoh stemming sederhana
words = ['pencarian', 'informasi', 'mencari', 'menemukan', 'relevan', 'pencari']
print("Kata Asli -> Hasil Stemming")
for word in words:
    stemmed = stemmer.stem(word)
    print(f"{word} -> {stemmed}")

# Output:
# pencarian -> cari
# informasi -> informasi
# mencari -> cari
# menemukan -> temu
# relevan -> relevan
# pencari -> cari
\end{lstlisting}

\textbf{Implementasi Preprocessing Lengkap:}
\begin{lstlisting}[language=python, style=python]
class TextPreprocessor:
    def __init__(self, language='indonesian'):
        self.language = language
        self.stemmer = StemmerFactory().create_stemmer()
        self.stopwords = set(stopwords.words(language))

    def preprocess(self, text):
        """Preprocessing lengkap untuk satu teks"""
        # 1. Normalisasi
        text = self.normalize_text(text)

        # 2. Tokenisasi
        tokens = word_tokenize(text)

        # 3. Filter token yang valid (hanya huruf, panjang > 2)
        tokens = [token for token in tokens if token.isalpha() and len(token) > 2]

        # 4. Stopword removal
        tokens = [token for token in tokens if token not in self.stopwords]

        # 5. Stemming
        tokens = [self.stemmer.stem(token) for token in tokens]

        return tokens

    def normalize_text(self, text):
        """Normalisasi teks"""
        # Case folding
        text = text.lower()

        # Hilangkan URL, email, angka
        text = re.sub(r'http\S+|www\S+|https\S+', '', text)
        text = re.sub(r'\S+@\S+', '', text)
        text = re.sub(r'\d+', '', text)

        # Hilangkan tanda baca
        text = re.sub(r'[^\w\s]', ' ', text)

        # Hilangkan spasi berlebih
        text = re.sub(r'\s+', ' ', text).strip()

        return text

    def preprocess_documents(self, documents):
        """Preprocessing untuk multiple dokumen"""
        return [self.preprocess(doc) for doc in documents]

# Contoh penggunaan
preprocessor = TextPreprocessor()

documents = [
    "Sistem pencarian informasi modern menggunakan berbagai teknik untuk meningkatkan relevansi hasil pencarian.",
    "Preprocessing teks adalah langkah penting dalam membangun sistem IR yang efektif dan akurat.",
    "Tokenisasi dan stemming membantu mengurangi variasi kata dalam dokumen sehingga pencarian lebih efisien."
]

processed_docs = preprocessor.preprocess_documents(documents)
for i, doc in enumerate(processed_docs, 1):
    print(f"Dokumen {i}: {doc}")
\end{lstlisting}

\addcontentsline{toc}{subsection}{Latihan}
\subsection*{Latihan}

\textbf{Latihan 1: Preprocessing Teks Sederhana}

Lakukan preprocessing pada teks berikut dan analisis hasilnya:

\begin{lstlisting}[language=python, style=python]
# Teks untuk diproses
sample_text = """
Universitas Syiah Kuala (UNSYIAH) adalah universitas negeri yang terletak di Banda Aceh, Aceh.
Universitas ini didirikan pada tahun 1961 dan merupakan universitas tertua di Aceh.
UNSYIAH memiliki berbagai fakultas seperti Fakultas Teknik, FMIPA, dan Fakultas Kedokteran.
Website: https://unsyiah.ac.id, Email: info@unsyiah.ac.id
"""

# TODO: Implementasikan preprocessing
# 1. Normalisasi teks
# 2. Tokenisasi
# 3. Hilangkan stopwords
# 4. Lakukan stemming
# 5. Hitung frekuensi kata

# Solusi:
from collections import Counter

preprocessor = TextPreprocessor()
processed_tokens = preprocessor.preprocess(sample_text)

print("Hasil preprocessing:")
print(processed_tokens)

# Hitung frekuensi kata
word_freq = Counter(processed_tokens)
print("\n5 kata paling sering muncul:")
for word, freq in word_freq.most_common(5):
    print(f"{word}: {freq}")
\end{lstlisting}

\textbf{Latihan 2: Perbandingan Sebelum dan Sesudah Preprocessing}

\begin{lstlisting}[language=python, style=python]
# Bandingkan ukuran vocabulary sebelum dan sesudah preprocessing
original_text = """
Sistem pencarian informasi membantu pengguna dalam mencari dan menemukan informasi yang relevan.
Pencarian informasi melibatkan berbagai teknik seperti indexing, ranking, dan retrieval.
Preprocessing teks adalah langkah penting untuk meningkatkan kualitas pencarian informasi.
"""

# Sebelum preprocessing
original_tokens = word_tokenize(original_text.lower())
original_vocab = set([token for token in original_tokens if token.isalpha()])

# Sesudah preprocessing
preprocessor = TextPreprocessor()
processed_tokens = preprocessor.preprocess(original_text)
processed_vocab = set(processed_tokens)

print(f"Vocabulary sebelum preprocessing: {len(original_vocab)} kata unik")
print(f"Vocabulary sesudah preprocessing: {len(processed_vocab)} kata unik")
print(f"Pengurangan: {len(original_vocab) - len(processed_vocab)} kata")
print(f"Persentase pengurangan: {((len(original_vocab) - len(processed_vocab)) / len(original_vocab)) * 100:.1f}%")

print("\nKata-kata yang dihilangkan:")
removed_words = original_vocab - processed_vocab
print(sorted(list(removed_words)))
\end{lstlisting}

\textbf{Latihan 3: Analisis Dampak Preprocessing}

\begin{lstlisting}[language=python, style=python]
# Analisis dampak setiap tahap preprocessing
def analyze_preprocessing_steps(text):
    """Analisis setiap tahap preprocessing"""
    print("=== ANALISIS TAHAP PREPROCESSING ===")
    print(f"Teks asli: {text}")
    print(f"Panjang teks asli: {len(text)} karakter")

    # Tahap 1: Normalisasi
    normalized = preprocessor.normalize_text(text)
    print(f"\n1. Setelah normalisasi: {normalized}")
    print(f"   Panjang: {len(normalized)} karakter")

    # Tahap 2: Tokenisasi
    tokens = word_tokenize(normalized)
    print(f"\n2. Setelah tokenisasi: {tokens}")
    print(f"   Jumlah token: {len(tokens)}")

    # Tahap 3: Filter token valid
    valid_tokens = [token for token in tokens if token.isalpha() and len(token) > 2]
    print(f"\n3. Setelah filter token: {valid_tokens}")
    print(f"   Jumlah token valid: {len(valid_tokens)}")

    # Tahap 4: Stopword removal
    filtered_tokens = [token for token in valid_tokens if token not in preprocessor.stopwords]
    print(f"\n4. Setelah stopword removal: {filtered_tokens}")
    print(f"   Jumlah token tersisa: {len(filtered_tokens)}")

    # Tahap 5: Stemming
    stemmed_tokens = [preprocessor.stemmer.stem(token) for token in filtered_tokens]
    print(f"\n5. Setelah stemming: {stemmed_tokens}")
    print(f"   Jumlah token final: {len(stemmed_tokens)}")

# Contoh penggunaan
test_text = "Pencarian informasi adalah proses mencari dan menemukan informasi yang relevan dari kumpulan dokumen."
analyze_preprocessing_steps(test_text)
\end{lstlisting}